\documentclass[12pt,letterpaper]{article}

\title{
\textbf{\LARGE OBD Bluetooth Interface and Android App} \\
\Large Project Proposal Document
}

\author{
\normalsize Patrick Landis (pal25) \\
\normalsize Schuyler Thompson (sdt16) (Current Team Lead) \\
\normalsize Advisor: Swarup Bhunia, PhD
}
	
\date{\normalsize January 28, 2013}
\usepackage[margin=1in]{geometry}
\usepackage{graphicx}
\usepackage{epstopdf}

\begin{document}
\maketitle

\newpage

\section{Technical}
\subsection{Problem Description}
Every car in the United States since 1996 is required, by law, to contain on-board diagnostics. This standard, codified in SAE J1939 as on-board diagnostics II (OBD-II), is used for everything from reading fuel levels, to checking engine warnings for diagnostic purposes. The OBD-II specification specifically requires cars to make available a port in every car that can be used to read messages. A device is then plugged into this port to read messages over various protocols implemented by the car manufacturer. Currently, technicians must manually plug in a device into this port in order to read OBD-II messages. Our project aims to add wireless functionality for OBD-II via Bluetooth to be interfaced with Android compatible devices. Furthermore this project aims to make communication cryptographically secure.


\subsection{Project Specifications}
This project has quite a few objectives because there are quite a few parts to it. In Appendix I is a diagram to show the purposed information exchange for this project.

\subsubsection{OBD-II Interface Component}
\begin{itemize}
	\item The biggest requirement for this project is the development of a circuit which will have the functionality to interface directly to a car's OBD-II port. 
	\item This component will need to be able to act as a transceiver with the OBD-II port.
	\item The component will also be responsible for encoding and decoding messages for each of the protocols allowed in the OBD-II standard which are:
	\begin{enumerate}
		\item SAE J/1850 (PWM/VPW)
		\item ISO 15765-2 / SAE J2284 (CAN)
		\item SAE J2411 (Single Wire CAN)
		\item ISO 9141-2/14230-4 (Keyword Protocol)
		\item Ford DCL (Mostly Depreciated) 
	\end{enumerate}
	\item This component will be required to communicate with the central microcontroller in via some still-undecided standard protocol.
	\item This component will be powered from the OBD-II port.
\end{itemize}

\subsubsection{Bluetooth Interface Component}
\begin{itemize}
	\item This component will be required to communicate with the central microcontroller in via some still-undecided standard protocol.
	\item This board will be responsible for communicating via Bluetooth to Android enabled devices.
	\item This component will be powered from the OBD-II port.
\end{itemize}

\subsubsection{Central MCU}
\begin{itemize}
	\item This MCU must be able to communicate with the OBD-II interface component via some still-undecided standard protocol.
	\item This MCU must be able to communicate with the Bluetooth interface component via some still-undecided standard protocol.
	\item This component will be powered from the OBD-II port.
	\item MCU will be used to control message flow of the system.
\end{itemize} 

\subsubsection{Bluetooth Functionality Android-side}
\begin{itemize}
	\item We will be writing an Android app to receive data from the OBD reader.
	\item We will use the Android SDK to implement the bluetooth functionality
	\item The app will be able to display all of the basic messages from the diagnostic part of OBD.
	\item This includes the fault codes, monitor readiness, and whether the malfunction indicator light (MIL) is on or off.
	\item The OBD fault codes follow a pattern of N0000, a letter followed by 4 numbers. The app will contain the database of all of the standard codes and their meanings for easy lookup. 
	\item  The monitor readiness is indicating whether the all of the monitors used by the OBD system are ready to make a reading. They are reset when the OBD system is reset, and if the monitors are set to not ready, the car will fail any emissions check performed by a DMV. 
	\item The MIL is the light commonly known as the check engine light. 
\end{itemize}

\subsubsection{Cryptographically Securing Messages}
We will use Bluetooth's Secure Simple Pairing for security. This is mandatory for all BT 2.1+ devices.

\begin{itemize}
	\item Use ``Just Works" form of pairing.
	\item The pairing key will be stored on the non-Android side of communication.
	\item Both devices must acknowledge the pairing request.
\end{itemize}

\subsection{Strategy for Achieving Objectives}

\subsection{Verification and/or Testing}

\section{Management}
\subsection{Team Complement}

\subsection{Project Management Plan}

\subsection{Budget}

\newpage

\section{Appendicies}
\subsection{Appendix I}
\begin{figure}[!ht]
\centering
\includegraphics[totalheight=15cm]{images/info_exchange_diagram.eps}
\caption{Information Flow Diagram}
\label{fig: image}
\end{figure}

\end{document}





