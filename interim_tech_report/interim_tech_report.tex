\documentclass[12pt,letterpaper]{article}

\title{
\textbf{\LARGE OBD Bluetooth Interface and Android App} \\
\Large Interim Technical Report Document
}

\author{
\normalsize Patrick Landis (pal25) \\
\normalsize Schuyler Thompson (sdt16) (Current Team Lead) \\
\normalsize Advisor: Swarup Bhunia, PhD
}
	
\date{\normalsize March 4, 2013}
\usepackage[margin=1in]{geometry}
\usepackage{graphicx}
\usepackage{epstopdf}
\usepackage{pgfgantt}
\usepackage{indentfirst}

\begin{document}

\maketitle
\newpage

\section*{Executive Summary}
A user with an Android based phone will be able to read and interpret diagnostic and operational data from their 1996 model year or newer car. On some models, they will be able to control some aspects of the car, such as starting the engine or rolling down the windows. This will be accomplished by having their Android device pair with a OBD (On-board Diagnostic) interface that is connected to the car over bluetooth, using a secure channel. 

\newpage

\section{Introduction}
Every car in the United States since 1996 is required, by law, to contain on-board diagnostics. This standard, codified in SAE J1939 as on-board diagnostics II (OBD-II). The OBD-II specification officially requires a few different things. The OBD-II specification requires that all cars make available data link connector and also place restrictions on where they can be placed. The specification defines a set of standard diagnostic trouble codes, the freeze frame (which is sort of like a black box from an airplane, in that it records when something goes wrong), and requirements for MIL lighting (which deals with the check engine light). In addition it also defines specific terminology and acceptable limits for emissions controls components. A device is then plugged into the data link port to read messages over various protocols implemented by the car manufacturer. \\

Currently, technicians must manually plug in a device into this port in order to read OBD-II messages. Furthermore the current solutions only have basic functionality and are primarily for technicians. Our project aims to add wireless functionality for OBD-II via Bluetooth to be interfaced with Android compatible devices. Furthermore this project aims to make communication cryptographically secure. By allowing for cryptographically secure wireless OBD-II messages to be transferred to fairly ubiquitous Android-enabled devices this project is geared to enthusiasts and curious electronic hackers.

There are numorous design problems being tackled here. One of the primary ones is the interfacing with all of the different protocols OBD-II supports. These each have very different electrical chararataristics, and need a separate circuit designed for each one of them. Another problem is keeping the power consuption low so the whole system can be powered off of a car's battery. This also presents the problem of dealing with a non-reliable power source. Car batteries' voltage is notoriously non-stable. The system needs to be able to deal with this problem and provide stable 5V and 3.3V rails. 

\section{Project Specifications}
This project has quite a few objectives because there are quite a few parts to it. This section will refer to Appendix I, a diagram which shows the information exchange, to clarify things.

\subsection{OBD-II Interface Component}
\begin{itemize}
	\item This component requires a circuit which will have the functionality to interface directly to a car's OBD-II port. 
	\item This component is required to plug into the data link connector specified in the OBD-II specification SAE J1962
	\item This component will need to be able to act as a transceiver with the OBD-II port.
	\item The component will also be responsible for encoding and decoding messages for each of the protocols allowed in the OBD-II standard which are:
	\begin{enumerate}
		\item SAE J/1850 (PWM/VPW)
		\item ISO 15765-2 / SAE J2284 (CAN)
		\item SAE J2411 (Single Wire CAN)
		\item ISO 9141-2/14230-4 (Keyword Protocol)
		\item Ford DCL (Mostly Depreciated) 
	\end{enumerate}
	\item This component will be required to communicate with the central microcontroller in via UART.
	\item This component will be powered from the OBD-II port.
	\item This component will be responsible for buffering messages until they can be sent to the MCU.
\end{itemize}

\subsection{Bluetooth Interface Component}
\begin{itemize}
	\item This component will be required to adhere to the Bluetooth 2.1+ specification.
	\item This component will be required to communicate with the central microcontroller in via UART.
	\item This board will be responsible for communicating via Bluetooth to Android enabled devices.
	\item This component will be powered from the OBD-II port.
	\item This component will be responsible for setting up Bluetooth communications with Android enabled devices.
	\item This component will be responsible for negotiating the shared key for cryptography purposes.
\end{itemize}

\subsection{Central Microcontroller}
\begin{itemize}
	\item This MCU must be able to communicate with the OBD-II interface component via UART.
	\item This MCU must be able to communicate with the Bluetooth interface component via UART.
	\item This component will be powered from the OBD-II port.
	\item MCU will be used to control message flow of the system.
	\item The MCU with be required to store a list of recent pairing keys to allow Android devices to auto-connect.
\end{itemize} 

\subsection{Bluetooth Functionality Android-side}
\begin{itemize}
	\item There will be writing an Android app to receive data from the OBD reader.
	\item It will use the Android SDK to implement the bluetooth functionality.
	\item The app will be able to display all of the basic messages from the diagnostic part of OBD.
	\item This includes the fault codes, monitor readiness, and whether the malfunction indicator light (MIL) is on or off.
	\item The OBD fault codes follow a pattern of N0000, a letter followed by 4 numbers. The app will contain the database of all of the standard codes and their meanings for easy lookup. 
	\item  The monitor readiness is indicating whether the all of the monitors used by the OBD system are ready to make a reading. They are reset when the OBD system is reset, and if the monitors are set to not ready, the car will fail any emissions check performed by a DMV. 
	\item The MIL is the light commonly known as the check engine light. 
	\item The app will use the Android Holo theme, and will follow the generally accepted design principals of Android UI design.
	\item The app will pair with one dongle at a time.
	\item The app will work with Android 4.1.

\end{itemize}

\subsection{Cryptographically Securing Messages}
This section doesn't describe any specific component of the system. This section describes functional requirements for the overall system to cryptographically secure wireless messages. The basis of this approach is to use Bluetooth's Secure Simple Pairing for security. This pairing protocol is mandatory for all BT 2.1+ devices.

\begin{itemize}
	\item Will use ``Just Works" form of pairing.
	\item The pairing key will be stored on the non-Android side of communication in the MCU.
	\item Both devices must acknowledge the pairing request.
\end{itemize}

\section{Methodology}
\subsection{Introduction}



\end{document}
