\documentclass[12pt,letterpaper]{article}

\title{
\textbf{\LARGE OBD Stuff and Things} \\
\Large Project Concept Document
}

\author{
\normalsize Patrick Landis (pal25) \\
\normalsize Schuyler Thompson (sdt16)
}
	
\date{\normalsize January 21, 2013}
\usepackage[margin=1in]{geometry}

\begin{document}
\maketitle

\section{Introduction}
Every car in the United States since 1996 is required, by law, to contain on-board diagnostics. This standard, codified in SAE J1939 as on-board diagnostics II (OBD-II), is used for everything from reading fuel levels, to checking engine warnings for diagnostic purposes. The OBD-II specification specifically requires cars to make available a port in every car that can be used to read messages. A device is then plugged into this port to read messages over various protocols implemented by the car manufacturer. Currently, technicians must manually plug in a device into this port in order to read OBD-II messages. Our project aims to add wireless functionality for OBD-II via Bluetooth to be interfaced with Android compatible devices. Furthermore this project aims to make communication cryptographically secure.


\section{Project Objectives}
This project has quite a few objectives

\subsection{Bluetooth Functionality Board-side}
Wireless!

\subsection{Bluetooth Functionality Android-side}
\begin{itemize}
	\item We will be writing an Android app to receive data from the OBD reader.
	\item We will use the Android SDK to implement the bluetooth functionality
	\item The app will be able to display all of the basic messages from the diagnostic part of OBD.
	\item This includes the fault codes, monitor readiness, and whether the malfunction indicator light (MIL) is on or off.
	\item The OBD fault codes follow a pattern of N0000, a letter followed by 4 numbers. The app will contain the database of all of the standard codes and their meanings for easy lookup. 
	\item  The monitor readiness is indicating whether the all of the monitors used by the OBD system are ready to make a reading. They are reset when the OBD system is reset, and if the monitors are set to not ready, the car will fail any emissions check performed by a DMV. 
	\item The MIL is the light commonly known as the check engine light. 
\end{itemize}

\subsection{Cryptographically Securing Messages}

\subsection{Power Requirements}
Needs to be able to be powered of car's power.
\end{document}





